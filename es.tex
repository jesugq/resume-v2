%------------------
%   Information
%------------------

\newcommand{\userfull}{
    Jesús González
}
\newcommand{\usernumb}{
    +52 271 108 5129
}
\newcommand{\usermail}{
    jesugq@gmail.com
}
\newcommand{\username}{
    jesugq
}
\newcommand{\userpost}{
    Puebla de Zaragoza
}

%------------------
%   Sections
%------------------

\newcommand{\projects}{
    Proyectos
}
\newcommand{\experience}{
    Experiencia
}
\newcommand{\education}{
    Educación
}
\newcommand{\skills}{
    Habilidades
}

%------------------
%   Education
%------------------

\newcommand{\timeengineer}{
    Ago 2015 - Jun 2020
}
\newcommand{\titleengineer}{
    Tecnológico de Monterrey
}
\newcommand{\subtitleengineer}{
    Campus Puebla
}
\newcommand{\descriptionengineer}{
    \docitemize{
        \item Ingeniero en Tecnologías Computacionales
    }
}

%------------------
%   Projects
%------------------

\newcommand{\timefinance}{
    Abr 2020
}
\newcommand{\titlefinance}{
    Finanza
    \doclink{https://github.com/jesugq/finance}{Github}
}
\newcommand{\subtitlefinance}{
    Python, Flask, SQLite
}
\newcommand{\descriptionfinance}{
    \docitemize{
        \item Desarollo de una aplicación web que simula la compra y venta de acciones.
        \item Almacenamiento de usuarios, sus acciones presentes y su historial de compra-venta en una base de datos relacional sencilla.
    }
}

\newcommand{\timecompiler}{
    Nov 2019
}
\newcommand{\titlecompiler}{
    Compilador
    \doclink{https://github.com/jesugq/compiler-simplified-syntax-tree}{Github}
}
\newcommand{\subtitlecompiler}{
    C, Flex, Bison
}
\newcommand{\descriptioncompiler}{
    \docitemize{
        \item Implementación de un compilador que reconozca gramáticas de Derivación a la Izquierda (1).
        \item Uso de árbol sintáctico simplificado junto a una tabla de símbolos y funciones para leer instrucciones.
    }
}

\newcommand{\timekubeet}{
    May 2018
}
\newcommand{\titlekubeet}{
    Kubeet
    \doclink{https://github.com/jesugq/kubeet-landing}{Github}
}
\newcommand{\subtitlekubeet}{
    Typescript, Angular 7, Firebase
}
\newcommand{\descriptionkubeet}{
    \docitemize{
        \item Mantenimiento de 13 componentes Angular encargados de la presentación de cursos en línea.
        \item Adición de módulos a un catálogo en línea de material de autoenseñanza, alojado en una base de datos no relacional.
    }
}

\newcommand{\timecrafter}{
    May 2018
}
\newcommand{\titlecrafter}{
    Crafter
    \doclink{https://github.com/jesugq/WCrafter-iOS}{Github}
}
\newcommand{\subtitlecrafter}{
    Swift
}
\newcommand{\descriptioncrafter}{
    \docitemize{
        \item Desarrollo del prototipo de una aplicación de monitoreo de autobuses en una sede para la plataforma iOS, que presenta al usuario una lista de autobuses crafter disponibles.
        \item Uso de los componentes visuales de XCode para crear una interfaz responsiva de 15 componentes.
    }
}

\newcommand{\timetree}{
    May 2017
}
\newcommand{\titletree}{
    Residencias
}
\newcommand{\subtitletree}{
    C
}
\newcommand{\descriptiontree}{
    \docitemize{
        \item Creación de una estructura de tipo Árbol AVL para almacenar datos de residencia de ejemplo.
        \item Funcionalidad para la inserción, recuperación, modificación y eliminación.
    }
}

\newcommand{\timeconnect}{
    Nov 2016
}
\newcommand{\titleconnect}{
    Conecta 4
}
\newcommand{\subtitleconnect}{
    Java, Swing
}
\newcommand{\descriptionconnect}{
    \docitemize{
        \item Programación de un juego Conecta 4 utilizando la librería gráfica Swing integrada en Netbeans.
        \item Controlado utilizandolas teclas en CLI y dando clic en GUI.
    }
}

%------------------
%   Experience
%------------------

\newcommand{\timetutor}{
    Ago 2018 - Nov 2018
}
\newcommand{\titletutor}{
    Tecnológico de Monterrey
}
\newcommand{\subtitletutor}{
    Tutor en Línea \docslash Puebla de Zaragoza
}
\newcommand{\descriptiontutor}{
    \docitemize{
        \item Seguimiento de 12 estudiantes durante su aprendizaje de Microsoft Word y Excel en un periodo de cuatro meses, generándoles una agenda semanal para facilitarles sus estudios.
        \item Evaluación y asesoría otorgada a los estudiantes a través de la plataforma PrepaNet.
    }
}

\newcommand{\timeit}{
    Ene 2016 - Abr 2016 
}
\newcommand{\titleit}{
    Tecnológico de Monterrey
}
\newcommand{\subtitleit}{
    Ayudante de TI \docslash Veracruz
}
\newcommand{\descriptionit}{
    \docitemize{
        \item Mantenimiento de ordenadores de trabajo Windows, en particular caché y archivos temporales.
        \item Instalación de Office, Adobe, y software de mantenimiento del ordenador.
    }
}



%------------------
%   Skills
%------------------

\newcommand{\titlelanguages}{
    Lenguajes
}
\newcommand{\subtitlelanguages}{
    Java, Python \docslash C \docslash MySQL, MongoDB \docslash Javascript, HTML, CSS
}

\newcommand{\titletechnologies}{
    Tecnologías
}
\newcommand{\subtitletechnologies}{
    Angular, React \docslash Git, Bash \docslash Latex, Microsoft Office, Google Suite
}

\newcommand{\titletheory}{
    Teoría
}
\newcommand{\subtitletheory}{
    Desarrollo Basado en Pruebas \docslash Estructuras de Datos, Diseño de Algoritmos \docslash Codigo Auto Documentado
}