%------------------
%   Information
%------------------

\newcommand{\userfull}{
    Jesús González
}
\newcommand{\usernumb}{
    +52 271 108 5129
}
\newcommand{\usermail}{
    jesugq@gmail.com
}
\newcommand{\username}{
    jesugq
}
\newcommand{\userpost}{
    Puebla, Puebla
}

%------------------
%   Sections
%------------------

\newcommand{\projects}{
    Proyectos
}
\newcommand{\experience}{
    Experiencia
}
\newcommand{\education}{
    Educación
}
\newcommand{\skills}{
    Habilidades
}

%------------------
%   Education
%------------------

\newcommand{\timeengineer}{
    Ago 2015 - Jun 2020
}
\newcommand{\titleengineer}{
    Tecnológico de Monterrey
}
\newcommand{\subtitleengineer}{
    Campus Puebla
}
\newcommand{\descriptionengineer}{
    \docitemize{
        \item Ingeniero en Tecnologías Computacionales
    }
}

%------------------
%   Projects
%------------------

\newcommand{\timefinance}{
    Abr 2020
}
\newcommand{\titlefinance}{
    Finanza
    \doclink{https://github.com/jesugq/finance}{Github}
}
\newcommand{\subtitlefinance}{
    Python, Flask, SQLite
}
\newcommand{\descriptionfinance}{
    \docitemize{
        \item Desarrollo de una aplicación web para la simulación de compra y venta de acciones.
        \item Almacenamiento realizado en una base de datos SQL Lite.
        \item Entre los datos presentes se encuentran los usuarios, sus acciones, y su historial de compra-venta.
    }
}

\newcommand{\timecompiler}{
    Nov 2019
}
\newcommand{\titlecompiler}{
    Compilador
    \doclink{https://github.com/jesugq/compiler-simplified-syntax-tree}{Github}
}
\newcommand{\subtitlecompiler}{
    C, Flex, Bison
}
\newcommand{\descriptioncompiler}{
    \docitemize{
        \item Diseño de un compilador que reconozca una Gramática de Derivación a la Izquierda.
        \item Utilización de la estructura de árbol sintáctico simplificado para ordenar las instrucciones otorgadas.
        \item Variables y funciones almacenadas en una tabla de símbolos, de estructura tipo tabla de hash. 
    }
}

\newcommand{\timekubeet}{
    May 2018
}
\newcommand{\titlekubeet}{
    Kubeet
    \doclink{https://github.com/jesugq/kubeet-landing}{Github}
}
\newcommand{\subtitlekubeet}{
    Typescript, Angular 7, Firebase
}
\newcommand{\descriptionkubeet}{
    \docitemize{
        \item Creación de trece componentes Angular de una aplicación de página única.
        \item Módulos encargados de presentar un catálogo de material autodidáctico con diferentes vistas.
        \item Recuperación de los datos a través de la base de datos Google Firebase.
    }
}

\newcommand{\timecrafter}{
    May 2018
}
\newcommand{\titlecrafter}{
    Crafter
    \doclink{https://github.com/jesugq/WCrafter-iOS}{Github}
}
\newcommand{\subtitlecrafter}{
    Swift
}
\newcommand{\descriptioncrafter}{
    \docitemize{
        \item Desarrollo de un prototipo de monitoreo de autobuses dentro de una empresa para la plataforma iOS.
        \item Uso de los componentes visuales en XCode para crear una interfaz responsiva de trece componentes.
    }
}

\newcommand{\timeresidences}{
    May 2017
}
\newcommand{\titleresidences}{
    Residencias
}
\newcommand{\subtitleresidences}{
    C
}
\newcommand{\descriptionresidences}{
    \docitemize{
        \item Creación de un sistema de alojamiento de datos de residencias sencillo.
        \item Traducción de la estructura dde datos tipo Árbol AVL para su guardado en un archivo de texto.
        \item Funcionalidad CRUD utilizando la consola, escribiendo y leyendo del archivo.
    }
}

\newcommand{\timeconnect}{
    Nov 2016
}
\newcommand{\titleconnect}{
    Conecta 4
}
\newcommand{\subtitleconnect}{
    Java, Swing
}
\newcommand{\descriptionconnect}{
    \docitemize{
        \item Programación de un juego visual de Conecta 4 utilizando la librería gráfica Swing en Netbeans.
        \item Seguimiento básico de turnos y reglas, operando con matrices. En versiones CLI y GUI.
    }
}

%------------------
%   Experience
%------------------

\newcommand{\timetutor}{
    Ago 2018 - Nov 2018
}
\newcommand{\titletutor}{
    PrepaNet
}
\newcommand{\subtitletutor}{
    Tutor en Línea \docslash Puebla de Zaragoza
}
\newcommand{\descriptiontutor}{
    \docitemize{
        \item Seguimiento de doce estudiantes durante su aprendizaje de Microsoft Word y Excel en un periodo de cuatro meses.
        \item Se les otorgó una agenda semanal diseñada en Photoshop para el facilitamiento de sus estudios.
        \item Evaluación y asesoría otorgada a través de la plataforma PrepaNet.
    }
}

\newcommand{\timeit}{
    Ene 2016 - Abr 2016 
}
\newcommand{\titleit}{
    Tecnológico de Monterrey
}
\newcommand{\subtitleit}{
    Ayudante de TI \docslash Veracruz
}
\newcommand{\descriptionit}{
    \docitemize{
        \item Mantenimiento de ordenadores de trabajo Windows, en peculiar la limpieza del caché y de archivos temporales.
        \item Instalación de aplicaciones office, Adobe y otros de mantenimiento del sistema.
    }
}



%------------------
%   Skills
%------------------

\newcommand{\titlelanguages}{
    Lenguajes
}
\newcommand{\subtitlelanguages}{
    Java, Python \docslash C \docslash MySQL, MongoDB \docslash Javascript, HTML \& CSS
}

\newcommand{\titletechnologies}{
    Tecnologías
}
\newcommand{\subtitletechnologies}{
    Angular, React \docslash Git, Bash \docslash Latex, Microsoft Office, Google Suite
}

\newcommand{\titlestudies}{
    Estudios
    
    \hspace{0pt}
    Relevantes
}
\newcommand{\subtitlestudies}{
    Codigo Auto Documentado \docslash Arquitectura de Software, Desarrollo Basado en Pruebas
    
    Bases de Datos \docslash Estructuras de Datos, Diseño de Algoritmos
}

%------------------
%   Cover
%------------------

\newcommand{\coverintro}{
    Estimado Director de Recursos Humanos,
}
\newcommand{\coverannexed}{
    Anexado a este correo encontrará mi currículum vitae. Busco aplicar a la posición Programador, ubicado en la plataforma TEC CSM.
}
\newcommand{\coverthanks}{
    Agradezco su tiempo, y la vacante disponible para formar parte de su equipo de desarrollo. Le deseo un excelente día, y me encuentro a sus órdenes.
}